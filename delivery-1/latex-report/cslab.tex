\documentclass[11pt]{article}

\usepackage[a4paper,
            bindingoffset=0.5cm,
            left=3cm,
            right=3cm,
            top=3cm,
            bottom=4cm,
            footskip=1.5cm]{geometry}
            
%
\usepackage[T1]{fontenc}
\usepackage{graphicx}
\usepackage{enumitem}
\usepackage{blindtext}
\usepackage{float}
%

\setcounter{secnumdepth}{3}

%%%%%%%%%%%%%%%%%%%%%
\title{\textbf{Critical Systems Lab - MESCC\\ Water Pumping Automated System}}
\date{ISEP, January 2024}
\author{Ricardo Mendes\\ 1201779
\and Arthur Gerbelli\\ 1220201}
%%%%%%%%%%%%%%%%%%%%%

\begin{document}

\maketitle              
\newpage
\tableofcontents
\newpage

%
\section{Introduction}

The current document, is the result of the work done during the first delivery of the CSLAB class.

The document is divided into three parts, each one of them focused on the evaluation topics: \textbf{requirements specification} documentation, \textbf{rationale for selected technology} and \textbf{list of physical sensor and/or actuator} used for the demo.

The system that we are modeling is a Water Pumping System (WPS) for two rainwater wells. These types of systems are essentially used to move water from a lower elevation to a higher one.

A Remote Status Station is also described in the document. Its main function is to give a level of observability of the WPS and to alert the \textit{maintenance team} for a possible failure.

There is also an additional feature. The status of the system should also be visible through a web server.


%%%
\section{Requirement Specification}

%%%
\subsection{Problem Domain}

%%%
\subsubsection{Stakeholder Needs}

Based on the system's description and some clarifications during the classes, we identified the following Stakeholder needs:

\begin{enumerate}[leftmargin=4em, font=\small, label=\textbf{SN-\arabic*}]
	\setlength\itemsep{.5em}
	\item 
		\begin{enumerate}[leftmargin=1.5em, font=\small, label=\textbf{.\arabic*:}]
		\setlength\itemsep{0em}	
		\item The water in the WPS must be pumped from a lower level to a higher one.
		\item Every WPS is an independent system; they don't have influence on each other.
		\end{enumerate}
	\item 
		\begin{enumerate}[leftmargin=1.5em, font=\small, label=\textbf{.\arabic*:}]
		\setlength\itemsep{0em}	
		\item The status of each element of the wet well needs to be displayed in a Remote Status Station (RSS). 
		\item The RSS must display the water level, the pump status, an alarme and a button to disable the alarm.
		\item The alarm must be ON when a problem in the system is identified.
		\end{enumerate}
		
	\item The status information must be accessible through a web page.
\end{enumerate}

\noindent
\textbf{SN-1} is WPS specific, \textbf{SN-2} is RSS specific and \textbf{SN-3} is Web Server specific.

%%%
\subsubsection{System Context}

For a better understanding of the system, we developed an external view, and so, identified external entities that do not belong to the system but interact with it. The following diagram is the output of this analysis:

\begin{figure}[H]
  \centering
  \includegraphics[width=300px]{../diagrams/system-context-01.png}
  \caption{System Context}
  \label{fig:System Context}
\end{figure}

Although some elements represented in the diagram are part of the WPS (sensor, control unit, pump and RSS), we divide them into subsystems with their own responsibilities and interactions. 

By decoupling the WPS responsibilities, we are simplifying it and turning the critical system requirements easier to grasp and model. 

The main identified external entities are: \textbf{Power Supply}, \textbf{Maintenance Team} and \textbf{Web Client}. 

Please notice the independent Power Supply for the Control Unit and the RSS as a way to deal with the critical requirements.

%%%
\subsubsection{Use Cases}

By analyzing the Stakeholder Needs, we can see that the main goal of the WPS is to control the water level inside the wet well. This goal can be captured in the model as the \textit{"Control Water Level"} use case of the \textit{Control Unit In Use} system context. 

\begin{figure}[H]
  \centering
  \includegraphics[width=300px]{../diagrams/use-cases-01.png}
  \caption{Use Case diagram}
  \label{fig:Use Case1}
\end{figure}

A closer look at the use case \textit{"Control Water Level"} gave rise to the below activity diagram. No alternative scenario was modeled.

\begin{figure}[H]
  \centering
  \includegraphics[width=300px]{../diagrams/use-case-activity-diagram-01.png}
  \caption{Use Case Activity diagram}
  \label{fig:Use Case2}
\end{figure}

%%%
\subsubsection{Measure of Effectiveness}

To be able to describe the performance of the system, some quantifiable characteristics of the WPS were identified:

\begin{itemize}
\setlength\itemsep{0em}
  \item Energy Consumption in \textit{kilowatts per hour}
  \item Wet Well Capacity in \textit{cubic meters};
  \item Water Inflow in \textit{liters per second};
  \item Water Outflow in \textit{liters per second}.
\end{itemize}

\begin{figure}[H]
  \centering
  \includegraphics[width=300px]{../diagrams/measure-of-effectiveness.png}
  \caption{Measure of Effectiveness diagram}
  \label{fig:MoE}
\end{figure}

As illustrated, each characteristic can be related to a specific element of the WPS.

%%%
\subsubsection{Functional Analysis}

For the functional analysis, we only focused our attention to the Control Unit; the most critical of the subsystems.

\begin{figure}[H]
  \centering
  \includegraphics[width=300px]{../diagrams/functional-analysis-wps.png}
  \caption{Functional Analysis diagram}
  \label{fig:Functional Analysis}
\end{figure}

For a reliable \textit{Water Level Control}, the Control Unit needs to be able to read data from the sensors and also get the current status of the water pumps.

This data is used to decide the correct behavior - turn on or off the water pumps - and publish the system's status to be consumed by the RSS.

The system's status also has the information that can trigger an alarm on the RSS side.

%%%
\subsubsection{Conceptual Subsystems}

After identifying conceptual subsystems during the functional analysis, we tried to capture the communication between them. This was achieved by pinpointing  the inputs and outputs of those subsystems.

\begin{figure}[H]
\centering
\begin{minipage}{.5\linewidth}
  \centering
  \includegraphics[width=150px]{../diagrams/conceptual-subsystem-communication-wps.png}
  \caption{Communication WPS}
  \label{fig:Conceptual Subsystem 1}
\end{minipage}%
\begin{minipage}{.5\linewidth}
  \centering
  \includegraphics[width=150px]{../diagrams/conceptual-subsystem-communication-rss.png}
  \caption{Communication RSS}
  \label{fig:Conceptual Subsystem 2}
\end{minipage}
\end{figure}

In the diagrams above, we analyzed the WSP Control Unit and the RSS.

%%%
\subsubsection{Traceability to Stakeholder}

%%%
\subsection{Solution Domain}

%%%
\subsubsection{System Requirements}

\begin{enumerate}[leftmargin=4em, font=\small, label=\textbf{SR-\arabic*}]
	\setlength\itemsep{.5em}
	\item 
		\begin{enumerate}[leftmargin=1.5em, font=\small, label=\textbf{.\arabic*:}]
		\setlength\itemsep{0em}
		\item While the water level is above the minimum level, WPS shall have a pump working.
		\item When the water level is below minimum level, WPS shall have all pumps stopped.
		\item If the water level is above the maximum level, then the WPS shall trigger an alarm at the Remote Status Station (RSS).
		\item A second pump shall be turned on only when the water level is above 2/3 the maximum water level.
		\item When only one pump is available, the maximum water level shall be reduced to 2/3.
		\end{enumerate}
	\item
		\begin{enumerate}[leftmargin=1.5em, font=\small, label=\textbf{.\arabic*:}]s
		\setlength\itemsep{0em}
		\item The status of all WPS shall be displayed on all RSS.
		\item If the alarm is ON, the button in the RSS shall only disable it.
		\end{enumerate}
	\item The status of all WPS shall be visible on one web page.

\end{enumerate}

%%%
\subsubsection{System Structure}

\begin{figure}[H]
  \includegraphics[width=300px]{../diagrams/system-structure.png}
  \caption{System Structure Diagram}
  \label{fig:System Structure Diagram}
\end{figure}

%%%
\subsubsection{System Behavior}

\begin{figure}[H]
  \includegraphics[width=300px]{../diagrams/state-machine-wps.png}
  \caption{State Machine}
  \label{fig:State Machine}
\end{figure}

%%%
%%%
%%%
\subsection{Analysis of safety and reliability}

\begin{enumerate}[leftmargin=4em, font=\small, label=\textbf{H-\arabic*:}]
	\setlength\itemsep{.5em}
	\item 
		\begin{itemize}
		\setlength\itemsep{0em}
        		\item \textbf{Description:} One of the pumps stops working.
		\item Cause: Mechanical problem.
    		\item Effect: Lost of redundancy and reduction of system performance.
    		\item \textbf{Mitigation:} Reduce the maximum water level to 2/3 and trigger alarm.
		\end{itemize}
	\item 
		\begin{itemize}
		\setlength\itemsep{0em}
    		\item \textbf{Description:} The two level sensors give contradictory readings, i.e. one above max and one below min.
		\item Cause: Sensor malfunction, connection issues.
    		\item Effect: Inapropriete system behavior. 
    		\item \textbf{Mitigation:} Choose a worst case or compare with the last reading to find the fault. Trigger alarm.
		\end{itemize}
	\item 
		\begin{itemize}
		\setlength\itemsep{0em}
    		\item \textbf{Description:} Power shortage.
		\item Cause: Multiple causes
    		\item Effect: Complete failure of the system.
    		\item \textbf{Mitigation:} RSS with independente power supply and trigger alarm.
		\end{itemize} 
	\item 
		\begin{itemize}
		\setlength\itemsep{0em}
    		\item \textbf{Description:} Both pumps stopped working.
		\item Cause: Mechanical problem.
    		\item Effect: Complete failure of the system.
    		\item \textbf{Mitigation:} Trigger alarm.
		\end{itemize} 
	\item 
		\begin{itemize}
		\setlength\itemsep{0em}
    		\item \textbf{Description:} RSS are not getting information from WPS.
		\item Cause: Connection issues or Messagem broker stoped working.
    		\item Effect: Wrong status readings.
    		\item \textbf{Mitigation:} Trigger alarm or remove broker as single point of failure by using protocols like DDS.
		\end{itemize} 
	\item 
		\begin{itemize}
		\setlength\itemsep{0em}
    		\item \textbf{Description:} RSS stops working.
		\item Cause: Malfunction.
    		\item Effect: Unknown WPS status.
    		\item \textbf{Mitigation:} Have redundancy by having multiple RSS and each one displaying all statuses from all WPS.
		\end{itemize} 
	\item 
		\begin{itemize}
		\setlength\itemsep{0em}
    		\item \textbf{Description:} A pump doesn't turn OFF when the water level in bellow minimum.
		\item Cause: Mechanical problem.
    		\item Effect: Pump overheating and complete failure.
    		\item \textbf{Mitigation:} Trigger alarm.
		\end{itemize} 
\end{enumerate}

\newpage
%%%
\section{Selected Technologies}

\begin{figure}[H]
  \includegraphics[width=300px]{../diagrams/deployment-diagram-WPS.jpg}
  \caption{Deployment diagram}
  \label{fig:Deployment Diagram}
\end{figure}

\newpage
%%%
\section{List of physical sensors/actuators}

\begin{figure}[H]
  \includegraphics[width=300px]{../diagrams/network-diagram-WPS.jpg}
  \caption{Network diagram}
  \label{fig:Network1 Diagram}
\end{figure}

\end{document}
