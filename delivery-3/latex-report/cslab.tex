\documentclass[11pt]{article}

\usepackage[a4paper,
            bindingoffset=0.5cm,
            left=3cm,
            right=3cm,
            top=3cm,
            bottom=4cm,
            footskip=1.5cm]{geometry}
            
%
\usepackage[T1]{fontenc}
\usepackage{graphicx}
\usepackage{enumitem}
\usepackage{blindtext}
\usepackage{float}
\usepackage{xcolor}
\usepackage{multirow}
\usepackage{minted}
 \usepackage{url}
 \usepackage{amsmath}
%

\setcounter{secnumdepth}{3}

%%%%%%%%%%%%%%%%%%%%%
\title{\textbf{Critical Systems Lab - MESCC\\ Water Pumping Automated System}}
\date{ISEP, January 2024, \textbf{Third Delivery}}
\author{Ricardo Mendes\\ 1201779
\and Arthur Gerbelli\\ 1220201}
%%%%%%%%%%%%%%%%%%%%%

\begin{document}

\maketitle              
\newpage
\tableofcontents
\newpage

%
\section{Introduction}

This is the last report that will summarize all the analysis and implementation of the final work.

We tried to focus on the conclusions and not bring back all the specificities of the previews analyses. 

Although the following chapter are tightly defined, the report should be read as an whole. 

%%%
\section{Requirement Specifications}

- ligar uma bomba de modo aleatorio de modo a nao corruer apenas uma.
- lista do que foi implementado

%%%
\subsection{System Requirements}

%%%
\subsection{System Structure and Traceability}


\newpage
%%%
\section{Implementation}

%%%
\subsection{CCSYA - Assembly}

- mostrar codigo da implementacao
- um dos comentarios no codigo de assembly está mal (beq)

%%%
\subsection{RTAES - Concorrency and Real Time Scheduling}

- mostrar codigo da implementacao
- onde mutex

    // REAL TIME WCET SIMULATION -------------------------------
    unsigned long start_time = millis();
    unsigned long finish_time;
    unsigned long duration;
    // REAL TIME WCET SIMULATION -------------------------------

    // REAL TIME WCET SIMULATION -------------------------------
    finish_time = millis();      
    duration = finish_time - start_time;
    Serial.print("TASK 3 - Execution Time[ms]: ");
    Serial.println(duration); 
    
    long next_release = 5000 - duration;
    if (next_release > 0) {
      vTaskDelay(next_release / portTICK_PERIOD_MS);
    } else {
      Serial.println("TRASK 3 - FAILED Deadline !!!");
    }
    // REAL TIME WCET SIMULATION -------------------------------
    

TASK 1 - Connection failed: 172.20.10.13
TASK 1 - Connection failed: 172.20.10.5
TASK 1 - Duration[ms]: 2009
TRASK 1 - FAILED Deadline !!!

connection timeout = 1s
that gives a lot of timeouts. specially with one of the sensors
mitigation.

----

TASK 3 - Duration[ms]: 40
TASK 1 - Duration[ms]: 917
TASK 2 - Duration[ms]: 0
TASK 1 - Duration[ms]: 658
TASK 1 - Duration[ms]: 754
TASK 3 - Duration[ms]: 770
TASK 2 - Duration[ms]: 0
TASK 1 - Duration[ms]: 649
TASK 1 - Duration[ms]: 799
TASK 3 - Duration[ms]: 29
TASK 2 - Duration[ms]: 0
TASK 1 - Duration[ms]: 176
TASK 1 - Duration[ms]: 997
TASK 2 - Duration[ms]: 0
TASK 1 - Duration[ms]: 187
TASK 3 - Duration[ms]: 157
TASK 1 - Duration[ms]: 990
TASK 2 - Duration[ms]: 0
TASK 1 - Duration[ms]: 1036

---



%%%
\subsection{COMCS - Communication}

- TCP pull strategy

- Arthur

%%%
\subsection{Prototype}

\subsubsection{Overview}

- diagram if implementation

\subsubsection{WPS}

- explicar a tabela dos inputs dos sensores e levantar alarm caso (A tabela agora é diferente da ultima vez)
- strategy for cluster of control unit nodes

\subsubsection{MQTT Broker}

- one topic per WPS

\subsubsection{RSS}

- reads broker every 1 sec
- sends alert after 10 sec without message
- subscribes to each WPS topics (iter)
- log who clicked the button and maintain LED on... because the system heals it self 

\subsubsection{Web Server}

%%%
\section{Team Work}


%%%%%%%%%%%%%%%%%%%%%
\newpage
\begin{thebibliography}{8}

\bibitem{c1}
 Espressif documentation: {\url{https://docs.espressif.com/projects/esp-idf/en/latest/esp32/api-reference/system/freertos_idf.html#id23/}}

\end{thebibliography}

\end{document}
